\documentclass[a4paper, oneside, 12pt]{memoir} 
%\paperwidth=225mm
%\paperheight=320mm

\usepackage{graphicx}
\graphicspath{ {/Volumes/ninjadrive/Dropbox/Maximes_Music/Marketing/maximes_LOGO/} }

\usepackage{nag}
\usepackage{fixltx2e}
\usepackage{microtype}
\usepackage{afterpage}

\usepackage{layout}
\usepackage{fullpage}

\usepackage{verbatim}

%% Suppress page numbering 
\usepackage{nopageno} 

%font
\renewcommand*\rmdefault{ppl}

%ragged right formatting
\usepackage{ragged2e}
\usepackage{everysel}
\usepackage{footmisc}
\newlength{\saveparindent}
\setlength{\saveparindent}{\parindent}

%url formatting
\usepackage{hyperref}

\begin{document} 

%\pdfpagewidth=225mm
%\pdfpageheight=320mm

%%%Title page

\Centering
\LARGE {Nicholas Charles Bochsa}\\
\vspace{20mm}

\Huge \textbf {MARCHE FUNÈBRE}\\
\textbf {from Messe de Requiem}\\
\vspace{10mm}
%\Large {21 January 1815}\\
\vspace{10mm}
\line(1,0){150}
\vspace{30mm}

\Large {Concert Band}\\
\vspace{5mm}
\normalsize {Flute 1.2, Oboe 1.2, Clarinet 1.2, Bass Clarinet, Bassoon 1.2}\\
{Alto Saxophone 1.2, Tenor Saxophone, Baritone Saxophone}\\
{Horn 1.2.3.4, Trumpet, Trombone, Euphonium, Tuba}\\
{Contrabass, Timpani, Tam-Tam}\\

\vspace{35mm}

\Large {1815}\\
Full Score / Partitur

\vspace*{\fill}

\large {Arranged by Craig Dabelstein}

\clearpage


%%%Imprint

\vspace*{\fill}
\begin{center}
	\includegraphics[width=14cm]{maximes_black_transparent.png}
\end{center}
\vfill

\RaggedRight
\setlength{\parindent}{\saveparindent}
\vspace*{\fill}
\tiny \copyright~ 2019 by Craig Dabelstein.\\
\medskip
ISMN-13: 979-0-720224-30-5\\
\medskip


Published by Maxime's Music.\\
\textsc {MM0232}\\
\clearpage

\section* {Notes}
\normalsize
After Napoleon was first defeated by a coalition of European powers in 1814, the coalition restored the throne of France to Louis XVIII on 6 April 1814. Louis XVIII returned to Paris on 24 April 1814 and the subsequent celebration of the Bourbon Restoration was the occasion for a Motet, by Bochsa, ``Composed for the celebration of the Apoth\'{e}ose of Louis XVI and the Happy Return of the Bourbons''.

After ``The Hundred Days,'' during which Napoleon attempted to regain control, another, much larger celebration was held on 15 January 1815 centered on the reburial of the remains of Louis XVI and Marie Antoinette. It was for this celebration that Bochsa and Cherubini composed Requiems in honor of Louis XVI.

The Bourbon Restoration ended with the July Revolution of 1830, a commemoration of which resulted in the government's commission of Berlioz' \textit{Symphony for Band}.  Thus we have as book-ends for the Bourbon Restoration two large-scale, important original compositions for large band.


\section*{The Requiem for Louis XVI and Marie Antoinette}

Louis XVI had been beheaded on 21 January 1793, as a victim of the French Revolution. But France had always been a country with a father-figure at the head of society and after the period during which Napoleon was responsible for the death of an entire generation of French young men during his European Wars, the public began to look back to the harmless old king, Louis XVI. This turn in sentiment resulted in the reburial of Louis and his wife in a more suitable location. And so on the anniversary of  Louis' beheading, on 21 January 1815, a great ceremony was held in Paris which featured two government commissioned Requiems, one by Cherubini and one by Bochsa.

The importance of this occasion can be seen in the fact that on the very same day in Vienna an identical ceremony to commemorate Louis XVI was held in St. Stefan's Cathedral, organized by Talleyrand as an official event during the Congress of Vienna. The music on this occasion was a Requiem by Sigismund Neukomm, with Salieri conducting. Neukomm, whose birthplace was Salzburg,\footnote{Just around the corner from a house I maintained there for two years.} wrote a number of large-scale compositions for band which are unknown today.


Charles Nicholas Bochsa (1789--1856) was the son of Charles Bochsa, an oboist and conductor of a French regimental band who later moved to Paris to become a publisher. The son, Charles Nicholas Bochsa, was a prodigy, performing a piano concerto in public at age seven, a flute concerto of his own composition at age eleven and the following year composing a ballet. As a student at the Paris Conservatoire he studied with Catel and M\'{e}hul and while still in the Conservatoire  he joined with Erard, the piano manufacturer, to invent the double action harp. For this instrument Bochsa produced a vast number of studies which are still used today.

In 1813 Napoleon appointed Bochsa as the official harpist to his court and in this same year Bochsa began to compose the first of seven works for the Op\'{e}ra-Comique. The \textit{le Journal des d\'{e}bates} of 16 September 1815 looked back over these stage works and found that Bochsa's music had ``warmth, dramatic truth and, as they say, youth.''\footnote{Quoted in Michel Faul, \textit{Nicolas-Charles Bochsa} (Le Vallier: Editions Delatour France, 2003), 17. Faul quotes many reviews from Bochsa's English residence.}

On becoming an extremely well-known musician in Paris, Bochsa, perhaps under the pressure of having to associate with very successful and wealthy persons, began to create various kinds of letters of credit, forging the signatures of a large number of people and institutions for the purpose of obtaining money from their private accounts. One contemporary found that Bochsa had stolen 760,000 francs. To escape a court order for his arrest, branding and years of hard labor, Bochsa fled to London.

In London, Bochsa, by nature a showman, introduced himself by organizing eye-catching concerts such as one at Covent Garden for 13 harps, an oratorio, \textit{Le Déluge universal}, for chorus, 14 harps and double orchestra and composed his only opera in English, \textit{A Tale of Other Times}. His most successful idea was to found, in 1823, the Royal Academy of Music, a school patterned after the Paris Conservatoire.  Soon, however, there were rumors of ``freedom taken with the code of conduct'' and Bochsa was forced out of the direction of the school in 1826.  

It was at this time that Bochsa began his association with Anna Rivi\`{e}re, a very talented soprano who became the wife of Sir Henry Bishop, known locally as ``the English Mozart'' and already famous as the composer of the song, ``Home, Sweet Home.''\footnote{From his opera, \textit{Clari} (1829).} They met and began to become close during her appearances with the King's Theatre and the Italian Opera House, where Bochsa had become Musical Director. She was rapidly becoming famous and Bochsa, to take advantage of this, eloped with her and began to accompany her on extensive recital tours throughout Europe, including the Scandinavian countries, Russia and Italy, where Bochsa was appointed Director of the Regio Teatro San Carlo.  

In 1847, the couple sailed for America and performed in New York, Boston, Washington, Baltimore, Richmond, Charleston, Savannah and New Orleans. While passing through America, a review of one of their recitals appeared in the \textit{American Review} for 1846, the reviewer found the singing of Anna Bishop to be rather cold and not from the heart. Of Bochsa the review was more complimentary.
\begin{quotation}
  Bochsa is another instrumental wonder. The harp in his hands is full of splendid effects; it is capable of infinite variety in power and quality of tone, full of delicacy and of lyric fire. His execution is wonderful, and the variety of his touch still more so. His hands wander all over the strings and produce sounding arpeggios, rapid sparkling passages above, and harmonics as pure and silvery as we may imagine to come from the golden-wired harps of the cherubims. Few, who never heard such playing, can be aware of the scope of the instrument in solos, or indeed of its peculiar effects in the hands of such a master, as an accompaniment to the voice.
\end{quotation}

\clearpage

In 1849 Bochsa and Anna, whom he introduced as his pupil, made a nine-month tour of Mexico and in a journal\footnote{\textit{Travels of Anna Bishop in Mexico, 1949,} published without an author's name by Charles Deal in Philadelphia in 1852.} she kept we learning some personal characteristics of Bochsa. Here we read that Bochsa, at age sixty, was rather well-known for the ``rotundity of his form,''\footnote{Ibid., 14, 42.} near-sighted, an imposing figure who spoke with ``startling emphasis.''\footnote{Ibid., 212, 147, 126.} Reviews of his performance in Mexico City indicate that even at this advanced age Bochsa remained a great harpist.\footnote{Ibid., 104ff.}

\begin{quotation}
  He is, incontestably, the greatest harpist ever listened to.
  
\textit{Trait d'Union}
\end{quotation}


\begin{quotation}
  Clear as the tones of a nightingale in his touch, he completely overrules every difficulty of this undocile instrument, and, by the power of his genius, draws from it such torrents of harmony as overwhelm the audience with delight and wonder.
  
\textit{Siglo XIX}
\end{quotation}


\begin{quotation}
  Bochsa's harp solo is dwelt upon, as a composition of the most exquisite brilliancy and a performance of incredible power and beauty.
  
\textit{La Moda}
\end{quotation}

While in Mexico City, Bochsa actually composed, in three days time, an \textit{Operatta buffo, El Ensayo,} to be sung in Spanish.  In its review of this performance, the paper \textit{El Monitor} reminded its readers that Bochsa was known not only as the ``Paganini of the Harp,'' but ``as a composer of great skill and fecundity.''
\begin{quotation}
  Our limit of space will not permit us to analyze, as carefully as we would, this inspiration of one of the most celebrated composers of the age. It is rich in ideas, piquant and original, and the instrumentation is performed with that thorough knowledge of the orchestra, possessed by Bochsa to so high a degree of mastery.
\end{quotation}



They then returned to North America, giving many recitals, but now Bochsa's health was beginning to fail. The newspaper \textit{Daily Alta California} posted a notice on 8 July 1855:
\begin{quotation}
  We understand the old composer and conductor is in a precarious state of health and is afraid he will never leave California. A great musical light goes out.
\end{quotation}


Nevertheless, by December 1855, they were sailing again, now for Sydney, Australia. Within a month of their arrival, Bochsa died. A long cort\`{e}ge of local musicians formed a procession to his burial place, performing the slow movement of Beethoven's \textit{Third Symphony} and marches taken from the works of Handel. His broken-hearted companion commissioned an elaborate tomb, which shows her lying at the base of a tree with a harp lying against it, in the Camperdown Cemetery in Sydney, and reading:

\begin{quotation}
\begin{center}
	Sacred\\
To the memory of\\
Nicholas Charles Bochsa, Esq.\\
Who died 6th January 1856\\
This monument is erected in sincere\\
Devotedness by his faithful friend and pupil\\
Anna Bishop\\
-----\\
Mourn him --- mourn his harp-strings broken\\
Never more shall float such music\\
None could sweep the lyre like him!
\end{center}
\end{quotation}


Half a world away and six months later, an Irish newspaper carried a brief obituary.
\begin{quotation}
  Mr. Elia's Record for this week announces the death, in Australia, of Signor Bochsa, a man who, had he possessed more conduct and less charlatanry, might have left a permanent name in the annals of music, and not merely in Europe an ephemeral reputation, which, for better or worse, had died out long before he himself had died. Signor Bochsa was an original and brilliant harpist, allowing for a certain flashy vulgarity of taste, which seemed to cleave to all the man's doings. Some of his music for his instrument, both solo and concerted, has fancy and well intended (or adroitly borrowed) ideas.
  
\textit{The Cork Examiner,} 6 June 1856
\end{quotation}



After Bochsa's death, Anna continued her life as a traveling artist, with concerts in Asia (where she was wrecked on Wake Island and stranded for three weeks), India, back to Australia and then to New York where she died in 1884.  

\clearpage

\section*{Performance Practice}

\section*{Nr. 1 Marche fun\`{e}bre}
\begin{itemize}
	\item[] Bar 4.  The tenuto means to play this bar slower to heighten the feelings and then return to \textit{a tempo} in the following bar.
	\item[] Bars 22 and 24, 52 and 54.  The cadences with the quarter-notes with a dot over the head should be thought of not as staccato, but as little accents.
	\item[] Bars 26, 61.  The half-note when followed by rests at this time were played as a quarter-note.
	\item[] Bar 29.  Having the melodic line in this bar only in the first clarinet, while everything else is \textit{fortissimo,} suggests that the clarinet parts were doubled or more, whereas the oboe and high flute parts were probably not doubled.
	\item[] Bar 46.  At this time a bar with 4 \textit{forte} marks was taken to mean: \textit{forte, piu forte, piu forte, piu forte.} In passages where the full ensemble level is \textit{forte,} repeated \textit{forte} symbols in a single voice within such a passage should be taken as \textit{piu forte} in that voice, which in effect is an accent not a change of dynamics.
	\end{itemize}
	


\bigskip
\Centering
David Whitwell\\
Austin, 2015\\

\newpage
\thispagestyle{empty}
\mbox{}

\end{document}